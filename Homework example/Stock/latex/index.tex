\hypertarget{index_intro_sec}{}\section{Introduction}\label{index_intro_sec}
Basic application that shows multiple time series in a chart (using Qt Q\+ML and C++) This program demonstrates the use of\+: \begin{DoxyItemize}
\item Qt Q\+Network\+Access\+Manager to make and receive http requests \item Design a basic U\+I/front-\/end using Qt Q\+ML \item Implement the backend logic in C++ \item Integration of C++ in Q\+ML (two-\/way communication between Qt Q\+ML and C++) ~\newline
 ~\newline
 ~\newline
\end{DoxyItemize}
\hypertarget{index_how_sec}{}\section{Working\+:}\label{index_how_sec}
The program will retrieve stock information from Alpha Vantage by using one of their A\+P\+Is \begin{DoxyItemize}
\item {\ttfamily \href{https://www.alphavantage.co/documentation/}{\tt https\+://www.\+alphavantage.\+co/documentation/}} \item {\ttfamily \href{https://www.alphavantage.co/query?function=TIME_SERIES_DAILY&symbol=MSFT&apikey=demo}{\tt https\+://www.\+alphavantage.\+co/query?function=\+T\+I\+M\+E\+\_\+\+S\+E\+R\+I\+E\+S\+\_\+\+D\+A\+I\+L\+Y\&symbol=\+M\+S\+F\+T\&apikey=demo}} \end{DoxyItemize}
The stock data retrieved, is J\+S\+ON object containing time series with the stock\textquotesingle{}s open, high, low prices. ~\newline
 ~\newline
 ~\newline
\hypertarget{index_gui_sec}{}\section{The G\+U\+I / Front end}\label{index_gui_sec}
The G\+UI is built using Q\+ML. It has three buttons and one chart view Three buttons shows the symbol of the Stock that we want to view Chart view shows each stock chart when the respective button is pressed  